\documentclass{article}
\usepackage{amsthm}
\usepackage{enumitem}
\usepackage{titlesec}
\usepackage{chngcntr} % for custom numbering

% ==============================
% THEOREM ENVIRONMENTS
% ==============================
% Theorems numbered by subsubsection
\newtheorem{definition}{Definition}[subsubsection]
\newtheorem{proposition}{Proposition}[subsubsection]
\newtheorem{example}{Example}[subsubsection]

% ==============================
% CUSTOM NUMBERING
% ==============================
\renewcommand\thesubsection{\thesection.\arabic{subsection}}      % e.g. 1.1
\renewcommand\thesubsubsection{\thesubsection.\Alph{subsubsection}} % e.g. 1.1.A
\newcounter{subsubsubsection}[subsubsection]
\renewcommand\thesubsubsubsection{\thesubsubsection.\arabic{subsubsubsection}} % e.g. 1.1.A.1

% Custom commands for Marx-style manual override
\newcommand{\customsubsubsection}[2]{%
  \subsubsection*{#1 -- #2}%
}

\newcommand{\subsubsubsection}[1]{%
  \refstepcounter{subsubsubsection}%
  \paragraph{\thesubsubsubsection\quad #1}%
}

% ==============================
% HEADINGS
% ==============================
\titleformat{\section}{\LARGE\bfseries}{\thesection}{1em}{}
\titleformat{\subsection}{\Large\bfseries}{\thesubsection}{1em}{}
\titleformat{\subsubsection}{\large\bfseries}{\thesubsubsection}{1em}{}
\titleformat{\paragraph}[runin]{\bfseries}{\thesubsubsubsection}{1em}{} 

% ==============================
% DOCUMENT
% ==============================
\begin{document}

% === TITLE PAGE ===
\begin{titlepage}
  \centering
  \vspace*{2cm}
  {\Huge\bfseries Capital: A Critique of Political Economy \par}
  \vspace{1cm}
  {\Large Karl Marx \& Friedrich Engels (1867)\par}
  \vfill
  {\large Summary by Joshua Nicholls\par}
  \vspace*{2cm}
\end{titlepage}

% === PREFACES ===
\section*{Volume 1: The Process of Production of Capital}

\section*{Prefaces}
\subsection*{1867: Marx’s thank you letter to Engels}
\subsection*{1867: Dedication to Wilhelm Wolff}
\subsection*{1867: Preface to the First German Edition (Marx)}
\subsection*{Preface to the French Edition (Marx)}
\subsection*{1873: Afterword to the Second German Edition (Marx)}
\subsection*{1875: Afterword to the French Edition (Marx)}
\subsection*{1883: Preface to the Third German Edition (Engels)}
\subsection*{1886: Preface to the English Edition (Engels)}
\subsection*{1890: Preface to the Fourth German Edition (Engels)}

% === CHAPTER 1 ===
\part{Commodities and Money}

\section{Ch. 1.1: The Commodities}

% ==============================
% SECTION 1.1.1
% ==============================
\customsubsubsection{1.1.1A}{The Two Factors of a Commodity: Use-value and Value}

\begin{definition}[Use-value]\label{def:1.1.1.1}
The usefulness of a thing.
\begin{itemize}[noitemsep]
  \item Realised only in use or consumption.
  \item Conditioned by the physical properties of the commodity.
  \item Independent of the amount of labour required to appropriate its useful qualities.
  \item Forms the material content of wealth, whatever its social form may be.
\end{itemize}
\end{definition}

\begin{definition}[Exchange-value]\label{def:1.1.1.2}
The quantitative relation in which use-values of one kind exchange for use-values of another kind.
\begin{itemize}[noitemsep]
  \item Appears as the proportion in which commodities exchange.
  \item A form of appearance of an underlying value common to all commodities.
\end{itemize}
\end{definition}

\begin{proposition}[Dual Character of the Commodity]\label{prop:1.1.1.1}
Every commodity has a dual nature:
\begin{itemize}[noitemsep]
  \item As a \textit{use-value}, determined by its utility.
  \item As a \textit{value}, determined by the amount of abstract human labour embodied in it.
\end{itemize}
\end{proposition}

\begin{definition}[Value]\label{def:1.1.1.3}
The common substance that manifests itself in the exchange relation of commodities.
\begin{itemize}[noitemsep]
  \item Value is human labour in the abstract.
  \item The magnitude of value is determined by the socially necessary labour time required for production.
  \item Socially necessary labour time = labour time required under normal conditions of production and with average degree of skill and intensity prevalent at the time.
\end{itemize}
\end{definition}

% ==============================
% SECTION 1.1.2
% ==============================
\customsubsubsection{1.1.2B}{The Two-fold Character of the Labour Embodied in Commodities}

\begin{proposition}[Two Aspects of Labour]\label{prop:1.1.2.1}
The labour represented in commodities possesses a two-fold character:
\begin{itemize}[noitemsep]
  \item \textbf{Concrete labour:} Produces use-values by virtue of its particular kind.
  \item \textbf{Abstract labour:} Creates value by virtue of being expenditure of human labour in the abstract.
\end{itemize}
\end{proposition}

\begin{definition}[Concrete Labour]\label{def:1.1.2.1}
Labour considered in its specific, useful form.
\begin{itemize}[noitemsep]
  \item Produces definite articles of use (e.g. tailoring, weaving).
  \item Differentiated by the kind of activity, its object, its means, and its result.
  \item Source of use-value.
\end{itemize}
\end{definition}

\begin{definition}[Abstract Labour]\label{def:1.1.2.2}
Labour considered as the expenditure of human labour-power in general.
\begin{itemize}[noitemsep]
  \item Measured in time.
  \item Forms the substance of value.
  \item Represents labour stripped of its concrete, useful character.
\end{itemize}
\end{definition}

\begin{proposition}[Unity of the Two Aspects]\label{prop:1.1.2.2}
The commodity-form makes visible the two-fold character of labour:
\begin{itemize}[noitemsep]
  \item The commodity is simultaneously a use-value and a value.
  \item This duality reflects the dual character of the labour embodied in it.
\end{itemize}
\end{proposition}

\subsubsubsection{1} Example of a finer point under 1.1.2B.

\end{document}
