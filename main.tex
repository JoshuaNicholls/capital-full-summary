\documentclass{article}
\usepackage{amsthm}
\usepackage{titlesec}
\usepackage{enumitem}
\usepackage{etoolbox} % for \pretocmd

% ============================
% FONT SIZES / FONTS
% ============================
% Define custom fonts for each level (Google Doc sizes scaled to LaTeX)
\newcommand{\volumefont}{\fontsize{26pt}{28pt}\bfseries}
\newcommand{\partfont}{\fontsize{20pt}{22pt}\bfseries}
\newcommand{\chapterfont}{\fontsize{16pt}{18pt}\bfseries}
\newcommand{\sectionfont}{\fontsize{14pt}{16pt}\bfseries}
\newcommand{\subsectionfont}{\fontsize{12pt}{14pt}\bfseries}
\newcommand{\paragraphfont}{\fontsize{11pt}{13pt}\bfseries}
\newcommand{\subparagraphfont}{\fontsize{11pt}{13pt}\itshape}


% ============================
% VOLUME
% ============================
\newcounter{volume}
\renewcommand{\thevolume}{\Roman{volume}}
\newcommand{\volumeheading}[1]{%
  \refstepcounter{volume}%
  \vspace{2ex}%
  {\centering \volumefont Volume \thevolume: #1\par}%
  \vspace{2ex}%
}

% ============================
% Part heading for Preface (numberless)
% ============================
\newcommand{\partpreface}[1]{%
  \vspace{2ex}%
  {\centering \partfont #1\par}% prints the heading in Part-sized font
  \vspace{1ex}%
}

% ============================
% Part heading with number (independent)
% ============================
\newcounter{partnum}                % counter for numbered parts
\renewcommand{\thepartnum}{\Roman{partnum}}
\newcommand{\partheading}[1]{%
  \refstepcounter{partnum}% increments part number
  \vspace{2ex}%
  {\centering \partfont Part \thepartnum: #1\par}% prints number + title
  \vspace{2ex}%
}

% ============================
% CHAPTER
% ============================
\newcounter{chapter}[volume]
\renewcommand{\thechapter}{\thevolume.\arabic{chapter}}
\newcommand{\chapter}[1]{%
  \refstepcounter{chapter}%
  \vspace{2ex}%
  {\chapterfont Chapter \thechapter: #1\par}%
  \vspace{2ex}%
}

% ============================
% SECTION => SUBPARAGRAPH
% ============================

% Section
\renewcommand{\thesection}{\thechapter.\arabic{section}}
\titleformat{\section}[block]
  {\sectionfont}
  {Section \thesection}
  {1em}{}
                       
% Subsection
\renewcommand{\thesubsection}{\thesection.\Alph{subsection}}
\titleformat{\subsection}[block]
  {\subsectionfont}
  {\thesubsection}
  {1em}{}

% Paragraph
\setcounter{secnumdepth}{5}   % allow numbering down to subparagraph
\renewcommand{\theparagraph}{\thesubsection.\arabic{paragraph}}
\titleformat{\paragraph}[block]
  {\paragraphfont}
  {\theparagraph}
  {1em}{}

% Subparagraph
\setcounter{secnumdepth}{5}   % allow numbering down to subparagraph
\renewcommand{\thesubparagraph}{\theparagraph.\alph{subparagraph}}
\titleformat{\subparagraph}[block]
  {\subparagraphfont}
  {\thesubparagraph}
  {1em}{}

% ============================
% Spacing (optional, adjust to taste)
% ============================
\titlespacing*{\section}{0pt}{6ex}{5ex}
\titlespacing*{\subsection}{0pt}{5ex}{4ex}
\titlespacing*{\paragraph}{0pt}{4ex}{3ex}
\titlespacing*{\subparagraph}{0pt}{3ex}{2ex}

% --- Theorem definition (neutral base) ---
\newtheorem{innerdef}{Definition}
\newtheorem{innerprop}{Proposition}
\newtheorem{innerex}{Example}
\newtheorem{innerremark}{Remark}

% Wrap them as environments
\newenvironment{definition}{\begin{innerdef}}{\end{innerdef}}
\newenvironment{proposition}{\begin{innerprop}}{\end{innerprop}}
\newenvironment{example}{\begin{innerex}}{\end{innerex}}
\newenvironment{remark}{\begin{innerremark}}{\end{innerremark}}

% --- Macro to reattach definition environment dynamically ---

\makeatletter
\newcommand{\theoremattach}[1]{%
  % Definition
  \renewcommand\theinnerdef{\csname the#1\endcsname.\arabic{innerdef}}%
  \setcounter{innerdef}{0}%
  % Proposition
  \renewcommand\theinnerprop{\csname the#1\endcsname.\arabic{innerprop}}%
  \setcounter{innerprop}{0}%
  % Example
  \renewcommand\theinnerex{\csname the#1\endcsname.\arabic{innerex}}%
  \setcounter{innerex}{0}%
  % Remark
  \renewcommand\theinnerremark{\csname the#1\endcsname.\arabic{innerremark}}%
  \setcounter{innerremark}{0}%
}

% Attach to each heading
\pretocmd{\section}{\theoremattach{section}}{}{}
\pretocmd{\subsection}{\theoremattach{subsection}}{}{}
\pretocmd{\subsubsection}{\theoremattach{subsubsection}}{}{}
\pretocmd{\paragraph}{\theoremattach{paragraph}}{}{}
\pretocmd{\subparagraph}{\theoremattach{subparagraph}}{}{}
\makeatother

% ==============================
% DOCUMENT
% ==============================
\begin{document}

% === TITLE PAGE ===
\begin{titlepage}
  \centering
  \vspace*{2cm}
  {\Huge\bfseries Capital: A Critique of Political Economy \par}
  \vspace{1cm}
  {\Large Karl Marx \& Friedrich Engels (1867)\par}
  \vfill
  {\large Summary by Joshua Nicholls\par}
  \vspace*{2cm}
\end{titlepage}

% === PREFACES ===
\volumeheading{The Process of Production of Capital}

\partpreface{Prefaces}

\section*{1867: Marx's thank you letter to Engels}
\section*{1867: Dedication to Wilhelm Wolff}
\section*{1867: Preface to the First German Edition (Marx)}
\section*{Preface to the French Edition (Marx)}
\section*{1873: Afterword to the Second German Edition (Marx)}
\section*{1875: Afterword to the French Edition (Marx)}
\section*{1883: Preface to the Third German Edition (Engels)}
\section*{1886: Preface to the English Edition (Engels)}
\section*{1890: Preface to the Fourth German Edition (Engels)}

\newpage

% === CHAPTER 1 ===
\partheading{Commodities and Money}

\chapter{The Commodities}

% ==============================
% SECTION 1.1.1
% ==============================
\section{The Two Factors of a Commodity: Use-value and Value}

\begin{definition}[Use-value]\label{def:1.1.1.1}
The usefulness of a thing.
\begin{itemize}[noitemsep]
  \item Realised only in use or consumption.
  \item Conditioned by the physical properties of the commodity.
  \item Independent of the amount of labour required to appropriate its useful qualities.
  \item Forms the material content of wealth, whatever its social form may be.
\end{itemize}
\end{definition}

\begin{definition}[Exchange-value]\label{def:1.1.1.2}
The quantitative relation in which use-values of one kind exchange for use-values of another kind.
\begin{itemize}[noitemsep]
  \item Appears as the proportion in which commodities exchange.
  \item A form of appearance of an underlying value common to all commodities.
\end{itemize}
\end{definition}

\begin{proposition}[Dual Character of the Commodity]\label{prop:1.1.1.1}
Every commodity has a dual nature:
\begin{itemize}[noitemsep]
  \item As a \textit{use-value}, determined by its utility.
  \item As a \textit{value}, determined by the amount of abstract human labour embodied in it.
\end{itemize}
\end{proposition}

\begin{definition}[Value]\label{def:1.1.1.3}
The common substance that manifests itself in the exchange relation of commodities.
\begin{itemize}[noitemsep]
  \item Value is human labour in the abstract.
  \item The magnitude of value is determined by the socially necessary labour time required for production.
  \item Socially necessary labour time = labour time required under normal conditions of production and with average degree of skill and intensity prevalent at the time.
\end{itemize}
\end{definition}

\begin{example}
1 coat = 20 yards of linen.
\begin{itemize}[noitemsep]
  \item Not due to physical properties of coat or linen.
  \item Both embody equal quantities of human labour.
\end{itemize}
\end{example}

\newpage

% ==============================
% SECTION 1.1.2
% ==============================
\section{The Two-fold Character of the Labour Embodied in Commodities}

\begin{proposition}[Two Aspects of Labour]\label{prop:1.1.2.1}
The labour represented in commodities possesses a two-fold character:
\begin{itemize}[noitemsep]
  \item \textbf{Concrete labour:} Produces use-values by virtue of its particular kind.
  \item \textbf{Abstract labour:} Creates value by virtue of being expenditure of human labour in the abstract.
\end{itemize}
\end{proposition}

\begin{definition}[Concrete Labour]\label{def:1.1.2.1}
Labour considered in its specific, useful form.
\begin{itemize}[noitemsep]
  \item Produces definite articles of use (e.g. tailoring, weaving).
  \item Differentiated by the kind of activity, its object, its means, and its result.
  \item Source of use-value.
\end{itemize}
\end{definition}

\begin{definition}[Abstract Labour]\label{def:1.1.2.2}
Labour considered as the expenditure of human labour-power in general.
\begin{itemize}[noitemsep]
  \item Measured in time.
  \item Forms the substance of value.
  \item Represents labour stripped of its concrete, useful character.
\end{itemize}
\end{definition}

\begin{proposition}[Unity of the Two Aspects]\label{prop:1.1.2.2}
The commodity-form makes visible the two-fold character of labour:
\begin{itemize}[noitemsep]
  \item The commodity is simultaneously a use-value and a value.
  \item This duality reflects the dual character of the labour embodied in it.
\end{itemize}
\end{proposition}

\newpage

% ==============================
% SECTION 1.1.3
% ==============================
\section{The Form of Value or Exchange-Value}

\subsection{Elementary or Accidental Form of Value}

\paragraph{The Two Poles of Expression of Value: Relative Form and Equivalent Form}

\begin{definition}[Relative Form of Value]
The form in which the value of one commodity is expressed in terms of another commodity.
\begin{itemize}[noitemsep]
    \item Provides the basis for the social recognition of value.
    \item Represents the “other” commodity in which value is expressed.
\end{itemize}
\end{definition}

\begin{definition}[Equivalent Form of Value]
The form in which a commodity serves as the equivalent for all other commodities.
\begin{itemize}[noitemsep]
    \item The commodity chosen to express the value of all others.
    \item Example: historically, gold became the universal equivalent.
\end{itemize}
\end{definition}

% ============================
% Section I.1.3 continued
% ============================

\paragraph{The Relative Form of Value} % 3A2

\subparagraph{The Nature and Import of this Form} % 3A2a

\begin{definition}[Nature of Relative Form]
The relative form expresses the value of one commodity in the body of another. 
It allows value to appear distinct from use-value, as abstract labour embodied in a material other than its own. 
\end{definition}

\begin{proposition}
No commodity can express its value in itself; it requires another commodity to serve as the mirror of its value.
\end {proposition}

\subparagraph{Quantitative Determination of Relative Value} % 3A2b

\begin{definition}[Quantitative Relation]
The relative form of value is not only qualitative but quantitative:
A commodity expresses its value as a definite proportion in which it exchanges with another.
\end{definition}

\begin{example}
20 yards of linen = 1 coat.  
This proportion reflects the equal labour-time embodied in both commodities.
\end{example}

\paragraph{The Equivalent Form of Value} % 3A3

\begin{definition}[Equivalent Form]
The equivalent form is the complementary pole of the relative form:
a commodity assumes the role of directly expressing the value of another.
\end{definition}

\begin{proposition}[Asymmetry of Value-Forms]
The relative and equivalent forms are mutually necessary, yet asymmetrical: 
a commodity can be in the relative form or the equivalent form, but not both simultaneously.
\end{proposition}

\begin{example}[Aristotle]
Aristotle observed that exchange implies equality, and equality implies a common substance.  
However, he did not identify labour as this substance, constrained by the Ancient Greek conception of labour.
Its slavery-based mode of production is predicated on the fundamental inequality between people and, by extension, their labour.
\end{example}

\paragraph{The Elementary Form of Value Considered as a Whole} % 3A4

\begin{proposition}[Unity of Elementary Form]
The elementary form unites relative and equivalent forms into a single relation, 
where the value of one commodity necessarily appears in another.
\end{proposition}

\begin{definition}[Two Poles of Value Expression]
The elementary form always involves two poles: the expressing commodity (relative) 
and the commodity that serves as its expression (equivalent).
\end{definition}

\subsection{Total or Expanded Form of Value} % 3B

\paragraph{The Expanded Relative Form of Value} % 3B1

\begin{definition}[Expanded Relative Form]
A commodity may express its value in a series of different commodities, 
making value appear as distinct from any one use-value.
\end{definition}

\paragraph{The Particular Equivalent Form} % 3B2

\begin{proposition}[Relativity of Equivalence]
In the expanded form, each commodity can serve as the equivalent for another.  
The role of equivalence is thus relative, shifting according to the relation.
\end{proposition}

\paragraph{Defects of the Total or Expanded Form of Value} % 3B3

\begin{proposition}[Limits of the Expanded Form]
The expanded form is endless and unwieldy: 
it requires infinite comparisons and never achieves a unified expression of value.
\end{proposition}

\subsection{The General Form of Value} % 3C

\paragraph{The Altered Character of the Form of Value} % 3C1

\begin{definition}[General Form of Value]
All commodities express their value in a single commodity, which assumes the role of general equivalent.
\end{definition}

\paragraph{Interdependent Development of Relative and Equivalent Forms} % 3C2

\begin{proposition}[Mutual Development]
The relative and equivalent forms develop together: 
as all commodities take the relative form, one commodity is forced into the role of universal equivalent.
\end{proposition}

\paragraph{Transition to the Money-Form} % 3C3

\begin{proposition}[From General Form to Money]
The general equivalent becomes fixed in a single commodity—historically, the precious metals—thus giving rise to the money-form.
\end{proposition}

\subsection{The Money-Form} % 3D

\begin{definition}[Money-Form]
The money-form is the culmination of the value-form process: 
a specific commodity (gold, silver) becomes the universal equivalent, 
and all commodities express their value in it.
\end{definition}

\begin{remark}
The money-form resolves the infinite regress of the total or expanded form, 
providing a single, concrete equivalent for the value of all commodities.
\end{remark}

\newpage

% ============================
% Section I.1.4
% ============================

\section{The Fetishism of Commodities and Its Secret} % I.1.4

\begin{definition}[Commodity Fetishism]
Commodity fetishism is the process whereby social relations between people 
take the form of relations between things. The commodity appears to have 
value in itself, rather than as an expression of human labour.
\end{definition}

\begin{proposition}[Illusion of Autonomy]
The commodity seems to possess value independently of its production. 
This obscures the fact that value arises from human labour, not from the 
object’s natural properties.
\end{proposition}

\subsection{The Enigmatic Character of the Commodity}

\begin{definition}[Socially Necessary Labour Time]
Value is determined by the average labour time required under normal 
conditions of production. This social fact is concealed within the 
commodity’s material form.
\end{definition}

\begin{remark}
The fetish character arises because labour is represented not directly, 
but through the commodity’s exchangeability.
\end{remark}

\subsection{The Historical Specificity of the Fetish}

\begin{proposition}[Fetishism as Historical Form]
Fetishism is not universal: it emerges only under capitalist conditions, 
where labour produces commodities for exchange rather than direct use.
\end{proposition}

\begin{example}[Pre-Capitalist Relations]
In feudal society, labour obligations are transparent as personal duties. 
Only under capitalism do these relations become masked by the autonomy 
of commodities.
\end{example}

\subsection{Analogies with Religion}

\begin{definition}[Religious Analogy]
Just as in religion the products of the human mind appear as autonomous 
beings, in commodity fetishism the products of human labour appear as 
independent bearers of value.
\end{definition}

\begin{remark}
The commodity is thus a “social hieroglyphic,” whose true meaning can be 
deciphered only by uncovering the labour hidden within it.
\end{remark}

\subsection{The Inversion of Subject and Object}

\begin{proposition}[Inversion Principle]
In fetishism, relations between people assume the character of relations 
between things, while things (commodities) appear to act as social agents.
\end{proposition}

\begin{example}[Market Dynamics]
In the marketplace, commodities confront each other as though they 
possess will and agency, while the actual producers remain hidden.
\end{example}

\subsection{The Secret of the Fetish}

\begin{definition}[Secret of Commodities]
The secret of commodity fetishism lies in the dual character of labour: 
concrete labour creates use-values, while abstract labour creates value.
\end{definition}

\begin{remark}
By grasping this duality, the apparent mystery of commodities is resolved: 
value is nothing but congealed human labour.
\end{remark}

\end{document}